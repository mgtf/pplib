% $Id: outils_dev.tex,v 1.1 2002/01/15 15:43:05 tieum Exp $
\textcolor{section}{\section{Outils de d�veloppement}}
  \textcolor{subsection}{\subsection{Indentation du code, commentaires, organisation des fichiers}}
   \textcolor{subsubsection}{\subsubsection{indentation du code}}
     On indente le code \textbf{de deux espaces}, sans tabulation.
   \textcolor{subsubsection}{\subsubsection{en-t�te des fonctions php}}
     Nous utilisons phpxref pour commenter notre code. Ce syst�me permet de g�n�rer automatiquement 
     une documentation en HTML de toutes les fonctions et variables utilis�es.
     Veuillez respecter cette syntaxe pour tout fichier et fonction que vous cr�ez dans cette librairie.
     \\
     Exemple de fonction :
     \begin{verbatim}
     ////
     // !Une courte description du r�le de la fonction.
     // Une description plus longue ou vous detaillez ce que la fonction
     // prend comme parametres, ce qu'elle retourne ... ...
     function boo(...) 
     {
       // deux espaces pour indenter
       ...
     }
     \end{verbatim} 
     Notez le `////' pour commencer la description de la fonction.
     Notez �galement le `!' pour commenter le resum� de la fonction.
   \textcolor{subsubsection}{\subsubsection{Repertoire de travail}}
     votre r�pertoire de travail est `\~~/public\_html' auquel vous acc�derez via l'url : \\
     http://et/\~~votre\_login/
   \textcolor{subsubsection}{\subsubsection{En-t�te des fichiers pour les fichiers php}}
     Voici une en-t�te standard des fichiers � respecter \textbf{imp�rativement}.
     Exemple du fichier template\_image.inc :
     \begin{verbatim}
     <?
     /* $Id: outils_dev.tex,v 1.1 2002/01/15 15:43:05 tieum Exp $ */
     // template_image.inc - Ensemble de fonctions permettant l'insertion d'image texte dynamique 
     // template_image.inc - � base de font truetype (.ttf) dans un template
     ...
     ...
     \end{verbatim}
     Le format est le suivant : 
     `// nom\_exact\_du\_fichier - description du r�le de ce fichier.'
 \textcolor{subsection}{\subsection{Nom de variables}} 
     Pour les nom de variables \textbf{global}, prenez un nom de variable \textbf{long et explicite}~\\
     Pour les nom de variables \textbf{local}, prenez un nom de variable \textbf{court}\\
 \textcolor{subsection}{\subsection{Fonctions}}
     Les fonctions doivent �tre courtes, et ne doivent faire \textbf{qu'une seule chose}.\\
     Elle doivent tenir en largeur dans une petite fenetre (ISO/ANSI : 80x24).
     Une fonction ne doit pas contenir plus de 5 � 7 variables locales.\\
 \textcolor{subsection}{\subsection{Commentaires}}
     Ne cherchez pas � expliquer comment marche votre fonction, mais plut�t \textbf{ce qu'elle fait}.
     Pr�ferez un commentaire dans l'en t�te de votre fonction plut�t qu'� l'int�rieur de son code.
 \textcolor{subsection}{\subsection{CVS}}
    \textcolor{subsubsection}{\subsubsection{Consid�rations g�n�rales}}
    CVS est un syst�me de contr�le de version qui permet:
    \begin{itemize}
     \item de contr�ler la version de chaque fichier.
     \item de savoir ce qui a �t� modifier par quelqu'un d'autre.
     \item de mettre en ligne une application en uploadant uniquement les fichiers modifi�s
           depuis la pr�c�dante mise en ligne.
     \item de travailler � distance en r�cup�rant les sources d'un projet.
     \item de g�n�rer des fichiers de patchs entre deux versions d'un projet.
    \end{itemize}
    Nous utilisons CVS. Si vous ne savez pas ce que c'est, consultez : 
    http://www.cvshome.org \\
    Lorsque vous cr�ez un nouveau fichier, ajoutez :\\
    `/* \$Id\$ */' (pour du C ou du PHP) ou `$<$!-- \$Id\$ --$>$' (pour du HTML)
    en-t�te de ce fichier.
    Cette ligne sera remplac�e par diff�rentes informations utiles lorsque vous ferez un `cvs commit' sur ce fichier.
    \textcolor{subsubsection}{\subsubsection{Organisation des modules}}
    Regardez la liste des modules CVS existants ainsi que leur description en faisant :\\
    \begin{verbatim}
      cvs co -c
    \end{verbatim}
 \textcolor{subsection}{\subsection{TODO}}
